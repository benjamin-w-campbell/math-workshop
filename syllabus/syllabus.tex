\documentclass[12pt]{article}

%% -----------------------------------------------------------------------------
%% document structure packages
%% -----------------------------------------------------------------------------

\usepackage{multicol}           % multiple column handling
\usepackage{lmodern}
\usepackage[utf8]{inputenc}
\usepackage[T1]{fontenc}
\usepackage{enumerate}          % package for improved enumeration
\usepackage{appendix}           % for more appendix options
\usepackage{verbatim}           % for extended verbatim functionality


%% -----------------------------------------------------------------------------
%% font and encoding related packages (non-math)
%% -----------------------------------------------------------------------------

\usepackage{ulem}               % for alternative underline
\usepackage[charter,cal=cmcal]{mathdesign} % DK addition

%% -----------------------------------------------------------------------------
%% math related packages
%% -----------------------------------------------------------------------------

\usepackage{mhsetup}
\usepackage[fixamsmath]{mathtools}


%% -----------------------------------------------------------------------------
%% graphics related packages
%% -----------------------------------------------------------------------------

\usepackage[all]{xy}        % for xy graphics
\usepackage{graphicx}       % for graphic images
\usepackage{wrapfig}        % for wrapping
\usepackage{subfigure}      % for subfigures
\usepackage{tikz}           % tikz


%% -----------------------------------------------------------------------------
%% table and figure related packages
%% -----------------------------------------------------------------------------

\usepackage{multirow}       % for multiple rows in tables
\usepackage{booktabs}       % for professional tables
\usepackage{dcolumn}        % for academic tables
\usepackage{rotating}       % for rotating tables and figures
\usepackage{pdflscape}      % for landscape tables
\usepackage{array}          % for additional table settings
\usepackage{longtable}      % for long tables
\usepackage{threeparttable} % for table footnotes


%% -----------------------------------------------------------------------------
%% biblatex package and options
%% -----------------------------------------------------------------------------

%\usepackage[                    % biblatex package for
%  bibstyle=authoryear,          % author year style in bibliography
%  citestyle=authoryear-comp,    % author year style in citations
%  sorting=nyt,                  % sort bib by name, year, title
%  hyperref=auto,                % enable hyperref if cite style allows
%  % maxcitenames=2,               % maximum names before et al.
%  maxnames=3,
%  isbn=false,
%  url=false,
%  doi=false,
%  eprint=false
%]{biblatex}                     % load
%
%\bibliography{library}          % set default reference database


% ------------------------------------------------------------------ %
% Page formatting                                                    %
% ------------------------------------------------------------------ %

\usepackage[top=1in,right=1in,left=1in,bottom=1in]{geometry}
\usepackage{parskip}
\usepackage{color}

% Fancy headers
\usepackage{fancyhdr}
\pagestyle{fancy}
\fancyhf{}
\renewcommand{\headrulewidth}{0.3pt}
\renewcommand{\footrulewidth}{0.3pt}
\headheight 15.0pt

% Formatting
\setlength{\itemsep}{1em}     % item separation
%\setlength{\bibitemsep}{1em}  % bib item separation


% --------------------------------------------------------------------------- %
% optional packages and other options                                         %
% --------------------------------------------------------------------------- %

% Original author: Jason W. Morgan, borrowing heavily from courses taught by
% William Minozzi. Other contributions from Drew Rosenberg, Daniel Kent, Caleb
% Leininger.
\usepackage[pdftex,
pdftitle={Syllabus: Math Workshop for Political Science},
pdfauthor={Benjamin W. Campbell},
pdfsubject={syllabus},
pdfkeywords={political science, syllabus, methods, statistics},
pdfproducer={pdfLaTeX},
plainpages=false,
letterpaper=true,
citecolor=black,
linkcolor=blue,
colorlinks=true]{hyperref}

% header
\lhead{Math Workshop (2023)}
\chead{}
\rhead{Campbell \& Zhang}
%\lfoot{Syllabus version 1.0}
\cfoot{}
\rfoot{\thepage}


% --------------------------------------------------------------------------- %
% macros                                                                      %
% --------------------------------------------------------------------------- %

% Centered rule of variable thickness and length
% USAGE: \flexrule{width}{thick}.
\newcommand{\flexrule}[2]{
  \begin{center}
    \rule{#1\textwidth}{#2}
  \end{center}
}

\renewcommand{\labelitemi}{$\bullet$}
\renewcommand{\labelitemii}{$\circ$}
\renewcommand{\labelitemiii}{$\blackdiamond$} % mathabx package
\renewcommand{\labelitemiv}{$\diamond$}


% --------------------------------------------------------------------------- %
% begin document                                                              %
% --------------------------------------------------------------------------- %
\begin{document}
% this makes emphasis normal
\normalem

% --------------------------------------------------------------------------- %
% begin content                                                               %
% --------------------------------------------------------------------------- %

\thispagestyle{empty}
\begin{center}
  {\huge\bfseries Math Workshop for Political Science} \\[0.5em]
  {\LARGE\bfseries The Ohio State University} \\[0.5em]
  {\large\bfseries Syllabus: Autumn 2023} \\

  \vspace{2.5em}

  \begin{minipage}{0.4\linewidth} \flushleft
    \begin{tabular}{ll}
      Instructor:         & Benjamin W. Campbell \\
      Email:              & \emph{campbell.1721}\emph{\small @}\emph{osu.edu} \\[1.0em]
      Teaching Assistant: & Liuya Zhang \\
      Email:              & \emph{zhang.11580}\emph{\small @}\emph{osu.edu}
    \end{tabular}
  \end{minipage}
  \hspace{0.14\linewidth}
  \begin{minipage}{0.44\linewidth} \flushleft
    \begin{tabular}{ll}
      Class location: & 2130 Derby Hall \\
      Class time:     & M--F, 9:00--11:30 EDT \\[1.0em]
      Office location & 2049N Derby Hall \\
      Office hours:   & M-F, 13:00--14:00 EDT
    \end{tabular}
  \end{minipage}
\end{center}

\vspace{0.5em}


\flexrule{1.0}{0.5pt}

\section*{Description}

The purpose of this workshop is to provide incoming first year Ph.D. students
with some fundamental skills in various mathematical techniques that are used in
political science, regardless of sub-specialty, and generally to prepare
students for the first-year methods sequence. The workshop is also open to
continuing students, who feel that they would gain from participating in the
course.

This year the course will begin on Monday, August 7th and run every weekday
until Friday, August 18th. Sessions are tenantively scheduled to run from
9:00---11:30 each day. In the past, there has also been an additional meeting on
Saturday. We hope this will not be necessary this year, though if we fall behind
in the material, we may need to reconsider.


\subsubsection*{Textbook}

Moore, W.H. and Siegel, D.A., 2013.
\emph{A Mathematics Course for Political and Social Research}.
Princeton University Press. (SM)

Wickham, H. and Grolemund, G., 2017. \emph{R for Data Science}. O’Reilly. (WG)


\subsubsection*{General resources}

\begin{itemize}
\item Another popular, but more technical, textbook is: Simon, Carl P. and
  Lawrence Blume., 1994, \emph{Mathematics for Economists}. Vol. 7. New York:
  Norton.
\item MIT Open Courseware (Mathematics):
  \url{http://ocw.mit.edu/courses/#mathematics}
\item Khan Academy: \url{http://www.khanacademy.org/}
\item Brightstorm: \url{http://www.brightstorm.com/math/}
\item MathTV: \url{http://www.mathtv.com/videos\_by\_topic}
\end{itemize}

% \newpage

\section*{Class Format}

The workshop will be taught in a lecture format. During each day's session we
will review the problem set from the previous day before covering the current
day's material. Before each session, it is expected that students will complete
the previous day's problem set and review the current day's course
materials. While the course will be presented as a lecture, student
participation is strongly encouraged. For each module, links to a series of
short videos, lecture notes, and a daily problem set will be provided.

The day's preparatory videos are available in the course's Github repository
\href{https://github.com/benjamin-w-campbell/math-workshop}{here}. Problem sets and
lecture notes will be posted immediately following the end of that day's
session.

There are nine problem sets, each of which will be due the following morning at
the start of class. Students should plan to spend 2--4 hours each day completing
these assignments and preparing for the following session.


\section*{R Introduction}

The compressed nature of this class makes it impossible to give students a
comprehensive introduction to all of the tools they will need in the first year
methods sequence. However, a brief introduction to R and computational social
science will be provided during the last 3 lectures.

Our R resource for these 3 lectures will be a notebook that we will work through together.
While the Wickham and Grolemund book is a fantastic resource for one's methods coursework,
it is also quite thorough and extensive. For a brief introduction and overview
of R this workshop will be helpful.


\newpage
\section*{Class schedule}

\subsubsection*{Day 1: Introduction, Pre-test, Notation and Definitions, and
Some Basic Mathematics}

\begin{itemize} \itemsep-0.35em
\item Definition of a variable and real number systems
\item Set notation and relationships
\item Definition of independent and dependent variables
\item Discussion of interval notation
\item Definitions of types of functions
\item Commutative, associative, and distributive laws
\item Concepts of inequality and absolute value
\item Exponent rules
\end{itemize}

\subsubsection*{Day 2: Some Basic Mathematics (II)}

\begin{itemize} \itemsep-0.35em
%\item \textbf{Reading}: SM (3-21, Ch.2, Ch.3) \& WG (Preface, Ch.1)
\item Summation and product operators
\item Factorials, permutations, and combinations
\item Solving equations, inequalities, and for roots
  \begin{itemize} \itemsep-0.35em
  \item Single and multiple variables
  \item Quadratic formula
  \item Factoring
  \end{itemize}
\item Logarithms and rules
\end{itemize}

\subsubsection*{Day 3: Linear Algebra (I)}

\begin{itemize} \itemsep-0.35em
%\item \textbf{Reading}: SM (275-288, 297-298, 304-309) \& WG (Ch.2)
\item Linear equations and linear systems
\item Method of elimination
\item Definition of matrices and vectors
\item Matrix operators
\item Transposes
\item Dot product and matrix multiplication
\item Matrix representation of systems of equations
\end{itemize}

\subsubsection*{Day 4: Linear Algebra (II)}

\begin{itemize} \itemsep-0.35em
%\item \textbf{Reading}: SM (289-297, 315-324) \& WG (Chs.3-4)
\item Linear dependence/independence
\item Properties of matrix operators
\item Definition of identity, zero, and idempotent matrices
\item Reduced row/row echelon form and solving linear systems of equations
  Gauss-Jordan Reduction/Elimination
\end{itemize}

\subsubsection*{Day 5: Linear Algebra (III)}

\begin{itemize} \itemsep-0.35em
%\item \textbf{Reading}: SM (298-300, 310-315) \& WG (Chs.5-6)
\item Inverses
\item Conditions for nonsingularity of matrix
\item Definition of matrix rank
\item Determinants
\item Matrix inversion
\item Trace of a matrix
\item Eigenvectors and eigenvalues%$\dagger$
% \item Cofactor expansion
\end{itemize}

\subsubsection*{Day 6: Linear Algebra (IV) and Calculus (I -- Introduction to
  Differentiation)}

\begin{itemize} \itemsep-0.35em
% \item \textbf{Reading}: SM (81-92, 96-99, 103-113, Ch.6) \& WG (Ch.7)
% \item Cramer’s rule and equation solution
\item Limits
\item The difference quotient
\item The derivative
\item Rules of differentiation for a function of one variable
\item Rules of differentiation involving two or more functions of the same
  variable
\end{itemize}

\subsubsection*{Day 7: Calculus (II -- More Differentiation)}

\begin{itemize} \itemsep-0.35em
% \item \textbf{Reading}: SM (113-114, 355-361) \& WG (Ch.8)
\item Derivative of exponential and log functions
\item Rules of differentiation involving functions with different
  variables
\item Partial differentiation
\item Comments on differentiability and continuity
\item Second and higher derivatives
\end{itemize}

\subsubsection*{Day 8: Calculus (III -- Optimization and Constrained Optimization)}

\begin{itemize} \itemsep-0.35em
% \item \textbf{Reading}: SM (Ch.8) \& WG (Ch.9)
\item Definition of optimum and extreme values
\item Relative maximum and minimum
\item Second-derivative test
\item Constrained optimization and Lagrange Multipliers
\item Quadratic approximation and Taylor series expansion%$\dagger$
\item \textbf{R Introduction}
\begin{itemize}
  \item Installing R and RStudio
  \item Scripts vs. terminal vs. notebooks
  \item Coding best practices
\end{itemize}
\end{itemize}

\subsubsection*{Day 9: Calculus (IV -- Integration)}

\begin{itemize} \itemsep-0.35em
% \item \textbf{Reading}: SM (133-143) \& WG (Ch.10)
\item Antidifferentiation
\item Areas and Riemann sums
\item Indefinite and definite integrals
\item Fundamental Theorem of Calculus
\item \textbf{R Basics}
\begin{itemize}
  \item Object types
  \item Data storage
  \item Functional programming
  \item Data visualization
\end{itemize}
\end{itemize}

\subsubsection*{Day 10: Calculus (V -- More Integration)}

\begin{itemize} \itemsep-0.35em
% \item \textbf{Reading}: SM (144-150, 362-374) \& WG (Ch.14)
\item Integration by substitution
\item Integration by parts
\item Brief discussion of improper integrals
\item Calculus on matrices: the general rules%$\dagger$
\item \textbf{Data Analysis in R}
\begin{itemize}
  \item Exploratory data analysis on real data
\end{itemize}
\end{itemize}


% --------------------------------------------------------------------------- %
% bibliography                                                                %
% --------------------------------------------------------------------------- %
% \clearpage
% \insertbib
\end{document}
